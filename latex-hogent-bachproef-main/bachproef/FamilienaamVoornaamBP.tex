%===============================================================================
% LaTeX sjabloon voor de bachelorproef toegepaste informatica aan HOGENT
% Meer info op https://github.com/HoGentTIN/latex-hogent-report
%===============================================================================

\documentclass[dutch,dit,thesis]{hogentreport}

% TODO:
% - If necessary, replace the option `dit`' with your own department!
%   Valid entries are dbo, dbt, dgz, dit, dlo, dog, dsa, soa
% - If you write your thesis in English (remark: only possible after getting
%   explicit approval!), remove the option "dutch," or replace with "english".

\usepackage{lipsum} % For blind text, can be removed after adding actual content

%% Pictures to include in the text can be put in the graphics/ folder
\graphicspath{{graphics/}}

%% For source code highlighting, requires pygments to be installed
%% Compile with the -shell-escape flag!
\usepackage[section]{minted}
%% If you compile with the make_thesis.{bat,sh} script, use the following
%% import instead:
%% \usepackage[section,outputdir=../output]{minted}
\usemintedstyle{solarized-light}
\definecolor{bg}{RGB}{253,246,227} %% Set the background color of the codeframe

%% Change this line to edit the line numbering style:
\renewcommand{\theFancyVerbLine}{\ttfamily\scriptsize\arabic{FancyVerbLine}}

%% Macro definition to load external java source files with \javacode{filename}:
\newmintedfile[javacode]{java}{
    bgcolor=bg,
    fontfamily=tt,
    linenos=true,
    numberblanklines=true,
    numbersep=5pt,
    gobble=0,
    framesep=2mm,
    funcnamehighlighting=true,
    tabsize=4,
    obeytabs=false,
    breaklines=true,
    mathescape=false
    samepage=false,
    showspaces=false,
    showtabs =false,
    texcl=false,
}

% Other packages not already included can be imported here

%%---------- Document metadata -------------------------------------------------
% TODO: Replace this with your own information
\author{Ernst Aarden}
\supervisor{Dhr. F. Van Houte}
\cosupervisor{Mevr. S. Beeckman}
\title[Optionele ondertitel]%
    {Titel van de bachelorproef}
\academicyear{\advance\year by -1 \the\year--\advance\year by 1 \the\year}
\examperiod{1}
\degreesought{\IfLanguageName{dutch}{Professionele bachelor in de toegepaste informatica}{Bachelor of applied computer science}}
\partialthesis{false} %% To display 'in partial fulfilment'
%\institution{Internshipcompany BVBA.}

%% Add global exceptions to the hyphenation here
\hyphenation{back-slash}

%% The bibliography (style and settings are  found in hogentthesis.cls)
\addbibresource{bachproef.bib}            %% Bibliography file
\addbibresource{../voorstel/voorstel.bib} %% Bibliography research proposal
\defbibheading{bibempty}{}

%% Prevent empty pages for right-handed chapter starts in twoside mode
\renewcommand{\cleardoublepage}{\clearpage}

\renewcommand{\arraystretch}{1.2}

%% Content starts here.
\begin{document}

%---------- Front matter -------------------------------------------------------

\frontmatter

\hypersetup{pageanchor=false} %% Disable page numbering references
%% Render a Dutch outer title page if the main language is English
\IfLanguageName{english}{%
    %% If necessary, information can be changed here
    \degreesought{Professionele Bachelor toegepaste informatica}%
    \begin{otherlanguage}{dutch}%
       \maketitle%
    \end{otherlanguage}%
}{}

%% Generates title page content
\maketitle
\hypersetup{pageanchor=true}

%%=============================================================================
%% Voorwoord
%%=============================================================================

\chapter*{\IfLanguageName{dutch}{Woord vooraf}{Preface}}%
\label{ch:voorwoord}

%% TODO:
%% Het voorwoord is het enige deel van de bachelorproef waar je vanuit je
%% eigen standpunt (``ik-vorm'') mag schrijven. Je kan hier bv. motiveren
%% waarom jij het onderwerp wil bespreken.
%% Vergeet ook niet te bedanken wie je geholpen/gesteund/... heeft

\lipsum[1-2]
%%=============================================================================
%% Samenvatting
%%=============================================================================

% TODO: De "abstract" of samenvatting is een kernachtige (~ 1 blz. voor een
% thesis) synthese van het document.
%
% Een goede abstract biedt een kernachtig antwoord op volgende vragen:
%
% 1. Waarover gaat de bachelorproef?
% 2. Waarom heb je er over geschreven?
% 3. Hoe heb je het onderzoek uitgevoerd?
% 4. Wat waren de resultaten? Wat blijkt uit je onderzoek?
% 5. Wat betekenen je resultaten? Wat is de relevantie voor het werkveld?
%
% Daarom bestaat een abstract uit volgende componenten:
%
% - inleiding + kaderen thema
% - probleemstelling
% - (centrale) onderzoeksvraag
% - onderzoeksdoelstelling
% - methodologie
% - resultaten (beperk tot de belangrijkste, relevant voor de onderzoeksvraag)
% - conclusies, aanbevelingen, beperkingen
%
% LET OP! Een samenvatting is GEEN voorwoord!

%%---------- Nederlandse samenvatting -----------------------------------------
%
% TODO: Als je je bachelorproef in het Engels schrijft, moet je eerst een
% Nederlandse samenvatting invoegen. Haal daarvoor onderstaande code uit
% commentaar.
% Wie zijn bachelorproef in het Nederlands schrijft, kan dit negeren, de inhoud
% wordt niet in het document ingevoegd.

\IfLanguageName{english}{%
\selectlanguage{dutch}
\chapter*{Samenvatting}
\lipsum[1-4]
\selectlanguage{english}
}{}

%%---------- Samenvatting -----------------------------------------------------
% De samenvatting in de hoofdtaal van het document

\chapter*{\IfLanguageName{dutch}{Samenvatting}{Abstract}}

\lipsum[1-4]


%---------- Inhoud, lijst figuren, ... -----------------------------------------

\tableofcontents

% In a list of figures, the complete caption will be included. To prevent this,
% ALWAYS add a short description in the caption!
%
%  \caption[short description]{elaborate description}
%
% If you do, only the short description will be used in the list of figures

\listoffigures

% If you included tables and/or source code listings, uncomment the appropriate
% lines.
%\listoftables
%\listoflistings

% Als je een lijst van afkortingen of termen wil toevoegen, dan hoort die
% hier thuis. Gebruik bijvoorbeeld de ``glossaries'' package.
% https://www.overleaf.com/learn/latex/Glossaries

%---------- Kern ---------------------------------------------------------------

\mainmatter{}

% De eerste hoofdstukken van een bachelorproef zijn meestal een inleiding op
% het onderwerp, literatuurstudie en verantwoording methodologie.
% Aarzel niet om een meer beschrijvende titel aan deze hoofdstukken te geven of
% om bijvoorbeeld de inleiding en/of stand van zaken over meerdere hoofdstukken
% te verspreiden!

%%=============================================================================
%% Inleiding
%%=============================================================================

\chapter{\IfLanguageName{dutch}{Inleiding}{Introduction}}%
\label{ch:inleiding}

De inleiding moet de lezer net genoeg informatie verschaffen om het onderwerp te begrijpen en in te zien waarom de onderzoeksvraag de moeite waard is om te onderzoeken. In de inleiding ga je literatuurverwijzingen beperken, zodat de tekst vlot leesbaar blijft. Je kan de inleiding verder onderverdelen in secties als dit de tekst verduidelijkt. Zaken die aan bod kunnen komen in de inleiding~\autocite{Pollefliet2011}:

\begin{itemize}
  \item context, achtergrond
  \item afbakenen van het onderwerp
  \item verantwoording van het onderwerp, methodologie
  \item probleemstelling
  \item onderzoeksdoelstelling
  \item onderzoeksvraag
  \item \ldots
\end{itemize}

\section{\IfLanguageName{dutch}{Probleemstelling}{Problem Statement}}%
\label{sec:probleemstelling}

Uit je probleemstelling moet duidelijk zijn dat je onderzoek een meerwaarde heeft voor een concrete doelgroep. De doelgroep moet goed gedefinieerd en afgelijnd zijn. Doelgroepen als ``bedrijven,'' ``KMO's'', systeembeheerders, enz.~zijn nog te vaag. Als je een lijstje kan maken van de personen/organisaties die een meerwaarde zullen vinden in deze bachelorproef (dit is eigenlijk je steekproefkader), dan is dat een indicatie dat de doelgroep goed gedefinieerd is. Dit kan een enkel bedrijf zijn of zelfs één persoon (je co-promotor/opdrachtgever).

\section{\IfLanguageName{dutch}{Onderzoeksvraag}{Research question}}%
\label{sec:onderzoeksvraag}

Wees zo concreet mogelijk bij het formuleren van je onderzoeksvraag. Een onderzoeksvraag is trouwens iets waar nog niemand op dit moment een antwoord heeft (voor zover je kan nagaan). Het opzoeken van bestaande informatie (bv. ``welke tools bestaan er voor deze toepassing?'') is dus geen onderzoeksvraag. Je kan de onderzoeksvraag verder specifiëren in deelvragen. Bv.~als je onderzoek gaat over performantiemetingen, dan 

\section{\IfLanguageName{dutch}{Onderzoeksdoelstelling}{Research objective}}%
\label{sec:onderzoeksdoelstelling}

Wat is het beoogde resultaat van je bachelorproef? Wat zijn de criteria voor succes? Beschrijf die zo concreet mogelijk. Gaat het bv.\ om een proof-of-concept, een prototype, een verslag met aanbevelingen, een vergelijkende studie, enz.

\section{\IfLanguageName{dutch}{Opzet van deze bachelorproef}{Structure of this bachelor thesis}}%
\label{sec:opzet-bachelorproef}

% Het is gebruikelijk aan het einde van de inleiding een overzicht te
% geven van de opbouw van de rest van de tekst. Deze sectie bevat al een aanzet
% die je kan aanvullen/aanpassen in functie van je eigen tekst.

De rest van deze bachelorproef is als volgt opgebouwd:

In Hoofdstuk~\ref{ch:stand-van-zaken} wordt een overzicht gegeven van de stand van zaken binnen het onderzoeksdomein, op basis van een literatuurstudie.

In Hoofdstuk~\ref{ch:methodologie} wordt de methodologie toegelicht en worden de gebruikte onderzoekstechnieken besproken om een antwoord te kunnen formuleren op de onderzoeksvragen.

% TODO: Vul hier aan voor je eigen hoofstukken, één of twee zinnen per hoofdstuk

In Hoofdstuk~\ref{ch:conclusie}, tenslotte, wordt de conclusie gegeven en een antwoord geformuleerd op de onderzoeksvragen. Daarbij wordt ook een aanzet gegeven voor toekomstig onderzoek binnen dit domein.
\chapter{\IfLanguageName{dutch}{Stand van zaken}{State of the art}}%
\label{ch:stand-van-zaken}

% Tip: Begin elk hoofdstuk met een paragraaf inleiding die beschrijft hoe
% dit hoofdstuk past binnen het geheel van de bachelorproef. Geef in het
% bijzonder aan wat de link is met het vorige en volgende hoofdstuk.

% Pas na deze inleidende paragraaf komt de eerste sectiehoofding.

Dit hoofdstuk bevat je literatuurstudie. De inhoud gaat verder op de inleiding, maar zal het onderwerp van de bachelorproef *diepgaand* uitspitten. De bedoeling is dat de lezer na lezing van dit hoofdstuk helemaal op de hoogte is van de huidige stand van zaken (state-of-the-art) in het onderzoeksdomein. Iemand die niet vertrouwd is met het onderwerp, weet nu voldoende om de rest van het verhaal te kunnen volgen, zonder dat die er nog andere informatie moet over opzoeken \autocite{Pollefliet2011}.

Je verwijst bij elke bewering die je doet, vakterm die je introduceert, enz.\ naar je bronnen. In \LaTeX{} kan dat met het commando \texttt{$\backslash${textcite\{\}}} of \texttt{$\backslash${autocite\{\}}}. Als argument van het commando geef je de ``sleutel'' van een ``record'' in een bibliografische databank in het Bib\LaTeX{}-formaat (een tekstbestand). Als je expliciet naar de auteur verwijst in de zin (narratieve referentie), gebruik je \texttt{$\backslash${}textcite\{\}}. Soms is de auteursnaam niet expliciet een onderdeel van de zin, dan gebruik je \texttt{$\backslash${}autocite\{\}} (referentie tussen haakjes). Dit gebruik je bv.~bij een citaat, of om in het bijschrift van een overgenomen afbeelding, broncode, tabel, enz. te verwijzen naar de bron. In de volgende paragraaf een voorbeeld van elk.

\textcite{Knuth1998} schreef een van de standaardwerken over sorteer- en zoekalgoritmen. Experten zijn het erover eens dat cloud computing een interessante opportuniteit vormen, zowel voor gebruikers als voor dienstverleners op vlak van informatietechnologie~\autocite{Creeger2009}.

Let er ook op: het \texttt{cite}-commando voor de punt, dus binnen de zin. Je verwijst meteen naar een bron in de eerste zin die erop gebaseerd is, dus niet pas op het einde van een paragraaf.

\lipsum[7-20]

%%=============================================================================
%% Methodologie
%%=============================================================================

\chapter{\IfLanguageName{dutch}{Methodologie}{Methodology}}%
\label{ch:methodologie}

%% TODO: In dit hoofstuk geef je een korte toelichting over hoe je te werk bent
%% gegaan. Verdeel je onderzoek in grote fasen, en licht in elke fase toe wat
%% de doelstelling was, welke deliverables daar uit gekomen zijn, en welke
%% onderzoeksmethoden je daarbij toegepast hebt. Verantwoord waarom je
%% op deze manier te werk gegaan bent.
%% 
%% Voorbeelden van zulke fasen zijn: literatuurstudie, opstellen van een
%% requirements-analyse, opstellen long-list (bij vergelijkende studie),
%% selectie van geschikte tools (bij vergelijkende studie, "short-list"),
%% opzetten testopstelling/PoC, uitvoeren testen en verzamelen
%% van resultaten, analyse van resultaten, ...
%%
%% !!!!! LET OP !!!!!
%%
%% Het is uitdrukkelijk NIET de bedoeling dat je het grootste deel van de corpus
%% van je bachelorproef in dit hoofstuk verwerkt! Dit hoofdstuk is eerder een
%% kort overzicht van je plan van aanpak.
%%
%% Maak voor elke fase (behalve het literatuuronderzoek) een NIEUW HOOFDSTUK aan
%% en geef het een gepaste titel.

\lipsum[21-25]



% Voeg hier je eigen hoofdstukken toe die de ``corpus'' van je bachelorproef
% vormen. De structuur en titels hangen af van je eigen onderzoek. Je kan bv.
% elke fase in je onderzoek in een apart hoofdstuk bespreken.

%\input{...}
%\input{...}
%...

%%=============================================================================
%% Conclusie
%%=============================================================================

\chapter{Conclusie}%
\label{ch:conclusie}

% TODO: Trek een duidelijke conclusie, in de vorm van een antwoord op de
% onderzoeksvra(a)g(en). Wat was jouw bijdrage aan het onderzoeksdomein en
% hoe biedt dit meerwaarde aan het vakgebied/doelgroep? 
% Reflecteer kritisch over het resultaat. In Engelse teksten wordt deze sectie
% ``Discussion'' genoemd. Had je deze uitkomst verwacht? Zijn er zaken die nog
% niet duidelijk zijn?
% Heeft het onderzoek geleid tot nieuwe vragen die uitnodigen tot verder 
%onderzoek?

\lipsum[76-80]



%---------- Bijlagen -----------------------------------------------------------

\appendix

\chapter{Onderzoeksvoorstel}

Het onderwerp van deze bachelorproef is gebaseerd op een onderzoeksvoorstel dat vooraf werd beoordeeld door de promotor. Dat voorstel is opgenomen in deze bijlage.

%% TODO: 
%\section*{Samenvatting}

% Kopieer en plak hier de samenvatting (abstract) van je onderzoeksvoorstel.

% Verwijzing naar het bestand met de inhoud van het onderzoeksvoorstel
%---------- Inleiding ---------------------------------------------------------

\section{Introductie}%
\label{sec:introductie}

In een tijdperk waarin cloudtechnologieën en snelle, efficiënte implementatieprocessen essentieel zijn geworden, onderzoekt deze bachelorproef de optimale inzet van 'Infrastructure as Code' (IaC) tools, specifiek Ansible en Terraform, binnen IT-infrastructuurbeheer. Deze studie richt zich op IT-professionals die actief betrokken zijn bij het beheer en de ontwikkeling van infrastructuur. Deze gerichte doelgroep omvat systeembeheerders, DevOps-ingenieurs en IT-managers die streven naar het verbeteren van hun infrastructuurbeheerprocessen. \\\\

De centrale onderzoeksvraag luidt: "Hoe kunnen Ansible en Terraform effectief worden geïntegreerd in infrastructuurbeheer om de efficiëntie en operationele capaciteit te maximaliseren?" Deze vraag is bedoeld om de praktische toepassingen, voordelen, en synergieën van Ansible en Terraform in detail te onderzoeken. \\\\

Het doel van deze studie is tweeledig. Ten eerste is het gericht op het creëren van een gedetailleerd rapport met aanbevelingen, dat concrete strategieën en best practices biedt voor de implementatie van deze IaC tools. Ten tweede streeft het onderzoek naar het ontwikkelen van een proof-of-concept, dat de effectiviteit van de geïdentificeerde strategieën in een realistische omgeving demonstreert. \\\\

Het succes van deze bachelorproef zal worden gemeten aan de hand van de relevantie en bruikbaarheid van de aanbevelingen voor de specifieke doelgroep. Het eindresultaat moet IT-professionals in staat stellen hun cloudinfrastructuur efficiënter en effectiever te beheren, wat uiteindelijk zal leiden tot verbeterde operationele prestaties en concurrentievoordeel binnen hun respectievelijke industrieën. \\\\

%---------- Stand van zaken ---------------------------------------------------

\section{State-of-the-art}%
\label{sec:state-of-the-art}

\subsection{Infrastructure as Code: een evolueren landschap}

Infrastructure as Code (IaC) heeft als doel om infrastructuur te beheren met behulp van code, waarbij de klassieke manuele manier van infrastructuurbeheer niet meer gevolgd wordt. Deze verschuiving wordt grotendeels gedreven door twee cruciale ontwikkelingen in het IT-landschap: de steeds groter wordende vraag naar snelle implementatie en de opkomst van cloudcomputing \autocite{Guerriero_2019}. \\\\

Binnen IT is er altijd een nood aan snelheid. Bedrijven willen zo snel mogelijk nieuwe releases of producten uitbrengen, om te voldoen aan de noden van de klanten en te overleven in een snel veranderend landschap. Infrastructure as Code is een hulpmiddel om aan deze nood te voldoen. Met behulp van templates is het niet meer noodzakelijk om de hele configuratie van de servers en besturingssystemen te wijzigen wanneer een nieuwe applicatie ontwikkeld of gereleased wordt. Als bijvoorbeeld enkele servers moeten geüpdated worden, is het veel tijdsintensiever om deze updates één voor één handmatig toe te passen. Infrastructure as Code maakt het mogelijk om deze systemen met een simple declaratieve of imperatieve syntax automatisch te ‘provisioneren’. \\\\

De verdere evolutie van virtualisatie en de cloud heeft ertoe geleid dat infrastructuur is geëvolueerd  naar software en data \autocite{Guerriero_2019}. Hierdoor is infrastructuur gemakkelijker te beheren en kunnen snel veranderingen op grotere schaal worden aangebracht. Systeembeheerders hebben nu de mogelijkheid om systemen te ontwikkelen die het beheer en de configuratie van infrastructuur makkelijker en dynamischer maakt \autocite{Johann_2017}. \\\\

\subsection{Ansible en Terraform}

Ansible en Terraform zijn twee voorbeelden van Infrastructure as Code tools. Ansible gebruikt eenvoudige en leesbare procedurele syntax om infrastructuur te configureren en servers te deployen met behulp van SSH \autocite{Geerling_2020}. Zogenaamde ‘playbooks’ worden gebruikt om een of meerdere servers tegelijkertijd op te zetten of te configureren. Het speciale hierbij is dat deze playbooks idempotent zijn. Dat betekent dat playbooks meerde keren uitgevoerd kunnen worden, maar er geen wijziging zal plaatsvinden totdat men effectief iets veranderd heeft. Ansible heeft als voordeel dat er geen additionele software op servers moet geïnstalleerd worden, omdat men werkt met SSH om alle infrastructuur te configureren. \\\\

Terraform is heel gelijkaardig aan Ansible en werd ontwikkeld door HashiCorp. Er wordt gebruik gemaakt van de declaratieve HashiCorp Configuration Language (HCL) of JSON om infrastructuur te configureren en beheren. Terraform is eveneens idempotent en gebruikt een 'state file' om veranderingen over tijd bij te houden, wat het gemakkelijker maakt om veranderingen te volgen of terug te draaien. Wanneer veranderingen aangebracht worden, creëert Terraform een plan om het gewenste doel te bereiken. \\\\

Hoewel Ansible en Terraform op het eerste zicht beiden IaC tools lijken, worden ze in de praktijk vaak samen gebruikt, omdat elke tool zijn eigen sterke en zwakke punten heeft. Terraform wordt voornamelijk gebruikt voor het beheer van infrastructuur, terwijl Ansible hoofdzakelijk wordt gebruikt voor het installeren en configuren van software wanneer de infrastructuur reeds is opgezet \autocite{Ninawe_2023} . Terraform fungeert dus meer als een orchestration tool, terwijl Ansible een configuratiemanagementtool is. \\\\

De uitdaging is nu om deze tools optimaal samen te gebruiken. Hoewel beide tools in essentie dezelfde functionaliteit bezitten, worden ze op een verschillende manier geïmplementeerd. Er zijn studies uitgevoerd naar de implementatie van zowel Terraform als Ansible, maar de conclusies zijn nooit consistent \autocite{Gurbatov_2022}. Het is belangrijk om best practices te ontwikkelen voor het gezamenlijk gebruik van Terraform en Ansible, waarbij Terraform meer wordt ingezet voor provisioning en Ansible beter is voor configuratiemanagement. Het is cruciaal om de sterke en zwakke punten van zowel Terraform als Ansible grondig te analyseren om zo een optimale strategie te ontwikkelen voor hun gecombineerde gebruik. Deze aanpak stelt ons in staat om weloverwogen keuzes te maken over welke tool het beste ingezet kan worden voor specifieke taken binnen het spectrum van provisioning en configuratiemanagement. \\\\

%---------- Methodologie ------------------------------------------------------
\section{Methodologie}%
\label{sec:methodologie}

Aanvankelijk zal een uitvoerige comparatieve literatuurstudie worden uitgevoerd om de voor- en nadelen van zowel Terraform als Ansible te identificeren. Deze studie is essentieel om een theoretische basis te leggen voor de daaropvolgende empirische onderzoeksfase. Vervolgens zal de implementatie van Terraform en Ansible getoetst worden via een Proof of Concept, gebruikmakend van virtuele machines binnen een gecontroleerde testomgeving. Het primaire doel van dit praktijkgerichte onderzoek is het identificeren van potentiële valkuilen van beide tools, alsook het beoordelen van hun vermogen om specifieke functionaliteiten uit te voeren, al dan niet op een meer effectieve wijze dan de andere tool. Deze aanpak stelt ons in staat om niet alleen de theoretische capaciteiten van beide systemen te begrijpen, maar ook hun praktische toepasbaarheid en efficiëntie in realistische IT-scenario's.

%---------- Verwachte resultaten ----------------------------------------------
\section{Verwacht resultaat, conclusie}%
\label{sec:verwachte_resultaten}

In het kader van dit onderzoek worden enkele hypothesen gesteld. Deze hypothesen zijn gebaseerd op analyse van de voorafgaande literatuur. \\\\

Ten eerste wordt verondersteld dat Ansible een hogere efficiëntie zal vertonen dan Terraform bij het uitvoeren van configuratietaken binnen bestaande infrastructuur. Om deze hypothese te toetsen, wordt voorgesteld om de benodigde tijd te meten die noodzakelijk is om configuratietaken uit te voeren met beide tools. Daarnaast zal de complexiteit van de implementatie op een kwalitatieve manier geëvalueerd worden door verschillende case studies en de eigen PoC te analyseren. \\\\

Vervolgens wordt verwacht dat Terraform een superieure tool is ten opzichte van Ansible wat betreft het provisioneren van infrastructuur. Deze hypothese wordt getest door het succespercentage van infrastructuurdeployments die worden uitgevoerd met beide tools te vergelijken met elkaar. \\\\

Daarnaast biedt dit onderzoek een grote meerwaarde voor bedrijven bij het efficiënter beheren van hun IT-infrastructuur. De verworven inzichten kunnen een concurrentievoordeel opleveren en de bedrijfsefficiëntie vergroten. Inzicht in de potentiële valkuilen en de uitdagingen geassocieerd met het gebruik van Terreform en Ansible zal bedrijven in staat stellen om risico’s bij de implementatie van hun infrastructuur te minimaliseren. \\\\

Het is echter van belang te benadrukken dat, hoewel de hypthesen zijn gebaseerd op uitgebreid vooronderzoek, de daadwerkelijke onderzoeksresultaten significant kunnen verschillen van de verwachte resultaten. In het geval dat de uit uiteindelijke resultaten significant afwijken van de vooropgestelde hypotheses, zal het onderzoek desondanks nog steeds van grote waarde zijn. De nieuwe inzichten en de ontwikkelde best practices die uit dit onderzoek voortvloeien dragen bij aan een meer geïnformeerde en strategische benadering van IT-infrastructuurbeheer binnen bedrijven. \\\\



%%---------- Andere bijlagen --------------------------------------------------
% TODO: Voeg hier eventuele andere bijlagen toe. Bv. als je deze BP voor de
% tweede keer indient, een overzicht van de verbeteringen t.o.v. het origineel.
%\input{...}

%%---------- Backmatter, referentielijst ---------------------------------------

\backmatter{}

\setlength\bibitemsep{2pt} %% Add Some space between the bibliograpy entries
\printbibliography[heading=bibintoc]

\end{document}
